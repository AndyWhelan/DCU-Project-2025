\documentclass{article}
\usepackage{amsmath,amsfonts}
\usepackage{array}
\usepackage{cancel}
\usepackage[usenames, dvipsnames]{color}
\usepackage{enumerate}
\usepackage{graphicx}
\usepackage{tabularx}
\usepackage{textcomp}
\usepackage{url}
\usepackage{verbatim}
\usepackage{xcolor}
\usepackage{wasysym}
\definecolor{linkborder}{RGB}{204,229,255}
\usepackage{hyperref}\hypersetup{linkbordercolor=linkborder}

\definecolor{vanish}{RGB}{224,224,224}
\renewcommand{\arraystretch}{2}

\title{Balanis Quick Reference}
\author{Andrew Whelan}
\date{}
\begin{document}

\maketitle

\tableofcontents

\newpage

\section*{Notation}
\small
\begin{tabular}{ m{4em} m{24em} m{4em} }
    \( \partial_t \) & \( \displaystyle \frac{\partial}{\partial t}\) & \\
    \( E \) & Electric Field Intensity & [V/m] \\
    \( H \) & Magnetic Field Intensity & [A/m] \\
    \( D \) & Electric Flux Density & [C/m\textsuperscript{2}] \\
    \( B \) & Magnetic Flux Density & [Wb/m\textsuperscript{2}] \\
    \( \displaystyle J_{c/d/i} \) & Electric Current Density \textsubscript{(conduction/displacement/impressed)} & [A/m\textsuperscript{2}] \\
    \( \displaystyle \mathfrak{M}_{d/i} \) & Magnetic Current Density \textsubscript{(displacement/impressed)} & [V/m\textsuperscript{2}] \\
    \( \displaystyle \rho_{e} \) & Electric Charge Density & [C/m\textsuperscript{2}] \\
    \( \displaystyle \rho_{m} \) & Magnetic Charge Density & [Wb/m\textsuperscript{2}] \\
\end{tabular}
\normalsize
\newpage
\section{Basics}
\subsection{Maxwell's Equations}
\subsubsection*{Differential Form}
    \begin{subequations}\label{eq:Maxwell}
        \begin{align}
            \text{Maxwell-Faraday} & : & \nabla \times E &= - \partial_t B \begingroup \textcolor{vanish}{- \mathfrak{M}_i } \endgroup \label{eq:MaxwellFaraday} \\
            \text{Amp\`ere-Maxwell} & : & \nabla \times H &= \ \ \partial_t D + J_{c} + J_{i} \label{eq:AmpereMaxwell} \\
            \text{Gauss} & : & \nabla \cdot D &= \ \ \rho_e \label{eq:GaussElectric} \\
            \text{Gauss (Magnetism)} & : & \nabla \cdot B &= \ \ 0 \begingroup \textcolor{vanish}{\quad = \rho_m} \endgroup \label{eq:GaussMagnetic}
        \end{align}
    \end{subequations}
\subsubsection*{Integral Form}
    \begin{subequations}\label{eq:MaxwellIntegral}
        \begin{align}
            \oint_C E \cdot dl &=  - \partial_t \iint_S B \cdot ds \begingroup \textcolor{vanish}{- \iint_S \mathfrak{M_i} \cdot ds} \endgroup \label{eq:MaxwellFaradayIntegral} \\
            \oint_C H \cdot dl &=  \partial_t \iint_S D \cdot ds + \iint_S ( J_c + J_i ) \cdot ds \label{eq:AmpereMaxwellIntegral} \\
            \oiint_S D \cdot ds &= \iiint_V \rho_e \cdot dv \label{eq:GaussElectricIntegral} \\
            \oiint_S B \cdot ds &= \iiint_V \rho_m \cdot dv \label{eq:GaussMagneticIntegral}
        \end{align}
    \end{subequations}
\subsection{Constitutive Relations} 
    \begin{subequations}\label{eq:Constitutive}
        \begin{align}
            D &= \epsilon E \label{eq:ConstitutiveElectric} \\ 
            B &= \mu H \label{eq:ConstitutiveMagnetic} \\
            J_c &= \sigma E \label{eq:ConstitutiveCurrentDensity}
        \end{align}
    \end{subequations}
\newpage
\section{P.E.C Field Derivation}
    \begin{enumerate}
        \item   Starting from Maxwell's equations \eqref{eq:Maxwell} and constitutive equations \eqref{eq:Constitutive}, transform to frequency domain to simplify curl equations:
                \begin{subequations}\label{eq:MaxwellFD}
                    \begin{align}
                        \nabla \times H &= J + j \omega D \label{eq:AmpereMaxwellFD} \\
                        \nabla \times E &= - j \omega B \label{eq:MaxwellFaradayFD}
                    \end{align}
                \end{subequations}
        \item   Introduce magnetic vector potential $A$ and electric potential $\phi$ defined by:
                \begin{subequations}\label{eq:Potentials}
                    \begin{align}
                        B &= \nabla \times A \label{eq:MagneticPotential} \\
                        E &= - \nabla \phi - j \omega A \label{eq:ElectricPotential}
                    \end{align}
                \end{subequations}
                and fix Lorenz gauge:
                \begin{equation} \label{eq:LorenzGaugeCondition}
                    \nabla \cdot A = - \frac{j \omega}{c^2} \phi
                \end{equation}
        \item   Apply the above with some vector identities to reduce Ampere-Maxwell equation \eqref{eq:AmpereMaxwellFD} to Helmholtz equation:
                \begin{enumerate}
                    \item Substitute the constitutive conditions \eqref{eq:Constitutive} and potentials \eqref{eq:Potentials} respectively to \eqref{eq:AmpereMaxwellFD}:
                        \begin{align*}
                            & \frac{1}{\mu} \nabla \times B = J + j \omega \epsilon E \\
                            \implies & \frac{1}{\mu} \nabla \times \nabla \times A = J - j \omega \epsilon \nabla \phi - \omega^2 \epsilon A
                        \end{align*}
                    \item Use the vector calculus identity $\nabla \times \nabla \times A = \nabla(\nabla \cdot A) - \nabla^2 A$, so
                        \begin{align*}
                            \frac{1}{\mu} \left( \nabla(\nabla \cdot A) - \nabla^2 A \right) &= J - j \omega \epsilon \nabla \phi - \omega^2 \epsilon A
                        \end{align*}
                    \item Apply the Lorenz gauge condition \eqref{eq:LorenzGaugeCondition}:
                        \begin{align*}
                            & -\frac{1}{\mu} ( \frac{j \omega}{c^2} \nabla \phi + \nabla^2 A ) = J - j \omega \epsilon \nabla \phi - \omega^2 \epsilon A \\
                            \implies & - \frac{1}{\mu} \nabla^2 A = J - \omega^2 \epsilon A \\
                            \implies & \nabla^2 A + k^2 A = -\mu J.
                        \end{align*}
                \end{enumerate}
        \item   Solve for vector potential using Green functions to get
                \begin{equation} \label{eq:GreenSol1}
                    A = \mu \int J(r) G(r, r') dr'
                \end{equation}
        \item   Plug into electric field formula \eqref{eq:ElectricPotential} in terms of potentials:
                \begin{align*}
                    & A = \frac{j}{\omega} ( \nabla \phi + E )
                \end{align*}
                Use the far field assumption on the scalar potential $\nabla \phi \to 0$ - this can be worked out to die off quickly asymptotically. Finally we arrive at
                \begin{align*}
                    & E \approx - \omega \mu j \int J(r) G(r, r') dr' \\
                    \implies & E \approx - \eta k j \int J(r) G(r, r') dr'
                \end{align*}

    \end{enumerate}

\end{document}
