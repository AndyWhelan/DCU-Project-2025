\documentclass{article}
\usepackage{amsmath,amsfonts}
\usepackage{array}
\usepackage{cancel}
\usepackage[usenames, dvipsnames]{color}
\usepackage{enumerate}
\usepackage{graphicx}
\usepackage{tabularx}
\usepackage{textcomp}
\usepackage{url}
\usepackage{verbatim}
\usepackage{xcolor}
\usepackage{wasysym}
\definecolor{linkborder}{RGB}{204,229,255}
\usepackage{hyperref}\hypersetup{linkbordercolor=linkborder}

\definecolor{vanish}{RGB}{224,224,224}
\renewcommand{\arraystretch}{2}

\title{Balanis Quick Reference}
\author{Andrew Whelan}
\date{}
\begin{document}

\maketitle
\thispagestyle{empty}
\setcounter{page}{0}

\tableofcontents

\newpage
\small
\section*{Notation}
\subsection*{Mathematical}
\begin{tabular}{ m{7em} m{24em} m{4em} }
    \( \partial_t \) & \( \displaystyle \frac{\partial}{\partial t}\) & \\
    \( j \) & Imaginary unit ( \( j^2 = -1 \) ) & \\
\end{tabular}
\subsection*{Microscopic Fields}
\addcontentsline{toc}{section}{Notation}
\begin{tabular}{ m{7em} m{24em} m{4em} }
    \( E \) & Electric Field Intensity & [V m\textsuperscript{-1}] \\
    \( H \) & Magnetic Field Intensity & [A m\textsuperscript{-1}] \\
\end{tabular}
\subsection*{Macroscopic Fields}
\begin{tabular}{ m{7em} m{24em} m{4em} }
    \( D \) & Electric Flux Density & [C m\textsuperscript{-2}] \\
    \( B \) & Magnetic Flux Density & [Wb m\textsuperscript{-2}] \\
\end{tabular}
\subsection*{Field Sources}
\begin{tabular}{ m{7em} m{24em} m{4em} }
    \( \displaystyle J_{c/d/i} \) & Electric Current Density \textsubscript{(conduction/displacement/impressed)} & [A m\textsuperscript{-2}] \\
    \( \displaystyle \mathfrak{M}_{d/i} \) & Magnetic Current Density \textsubscript{(displacement/impressed)} & [V m\textsuperscript{-2}] \\
    \( \displaystyle \rho_{e} \) & Electric Charge Density & [C m\textsuperscript{-2}] \\
    \( \displaystyle \rho_{m} \) & Magnetic Charge Density & [Wb m\textsuperscript{-2}] \\
\end{tabular}
\subsection*{Constitutive Parameters}
\begin{tabular}{ m{7em} m{24em} m{4em} }
    \( \displaystyle \epsilon \) & Permittivity & [F m\textsuperscript{-1}] \\
    \( \displaystyle \mu \) & Permeability & [H m\textsuperscript{-1}] \\
    \( \displaystyle \sigma \) & Conductivity & [S m\textsuperscript{-1}] \\
\end{tabular}
\subsection*{Waves}
\begin{tabular}{ m{7em} m{24em} m{4em} }
    \( \displaystyle f \) & Frequency & [s\textsuperscript{-1}] \\
    \( \displaystyle k \)\textsuperscript{\eqref{eq:wavenumber}} & Wavenumber & [m\textsuperscript{-1}]  \\
    \( \displaystyle \alpha \)\textsuperscript{\eqref{eq:attenuationConstant}} & Wave Attenuation & [m\textsuperscript{-1}]  \\
    \( \displaystyle \lambda := \frac{2 \pi}{k} \) & Wavelength & [m] \\
    \( \displaystyle \omega := 2 \pi f \) & Angular Frequency & [s\textsuperscript{-1}] \\
    \( \displaystyle \eta := \sqrt{\frac{j \omega \mu}{\sigma + j \omega \epsilon}} \) & Wave Impedance & [S\textsuperscript{-1}] \\
    \( \displaystyle v_g := \partial_k \omega \) & Group velocity of wave (envelope velocity) & [ms\textsuperscript{-1}]  \\
    \( \displaystyle v_p := \frac{\omega}{k} \) & Phase velocity of wave (peak/trough velocity) & [ms\textsuperscript{-1}]  \\
\end{tabular}
\normalsize
\newpage
\section{Basics}
\subsection{Maxwell's Equations}
\subsubsection*{Differential Form}
    \begin{subequations}\label{eq:Maxwell}
        \begin{align}
            \text{Maxwell-Faraday} & : & \nabla \times E &= - \partial_t B \begingroup \textcolor{vanish}{- \mathfrak{M}_i } \endgroup \label{eq:MaxwellFaraday} \\
            \text{Amp\`ere-Maxwell} & : & \nabla \times H &= \ \ \partial_t D + J_{c} + J_{i} \label{eq:AmpereMaxwell} \\
            \text{Gauss} & : & \nabla \cdot D &= \ \ \rho_e \label{eq:GaussElectric} \\
            \text{Gauss (Magnetism)} & : & \nabla \cdot B &= \ \ 0 \begingroup \textcolor{vanish}{\quad = \rho_m} \endgroup \label{eq:GaussMagnetic}
        \end{align}
    \end{subequations}
\subsubsection*{Integral Form}
    \begin{subequations}\label{eq:MaxwellIntegral}
        \begin{align}
            \oint_C E \cdot dl &=  - \partial_t \iint_S B \cdot ds \begingroup \textcolor{vanish}{- \iint_S \mathfrak{M_i} \cdot ds} \endgroup \label{eq:MaxwellFaradayIntegral} \\
            \oint_C H \cdot dl &=  \partial_t \iint_S D \cdot ds + \iint_S ( J_c + J_i ) \cdot ds \label{eq:AmpereMaxwellIntegral} \\
            \oiint_S D \cdot ds &= \iiint_V \rho_e \cdot dv \label{eq:GaussElectricIntegral} \\
            \oiint_S B \cdot ds &= 0 \begingroup \textcolor{vanish}{= \iiint_V \rho_m \cdot dv} \endgroup \label{eq:GaussMagneticIntegral}
        \end{align}
    \end{subequations}
\subsection{Constitutive Relations} 
    \begin{subequations}\label{eq:Constitutive}
        \begin{align}
            D &= \epsilon E \label{eq:ConstitutiveElectric} \\ 
            B &= \mu H \label{eq:ConstitutiveMagnetic} \\
            J_c &= \sigma E \label{eq:ConstitutiveCurrentDensity}
        \end{align}
    \end{subequations}
\subsection{Boundary Conditions}

\scriptsize
\begin{tabular}{ | m{6.5em} || m{7.9em} | m{7.5em} | m{7.7em} | m{7.9em} | }
    \hline
    & \textbf{General} & \textbf{Finite \( \sigma \), no source/charge} & \textbf{Medium 1 PEC} & \textbf{Medium 1 PMC} \\
    \hline\hline
    \( E_{\parallel} := n \times E \) & \( E_{\parallel 2} - E_{\parallel 1} = -\mathfrak{M}_s \) & \( E_{\parallel 2} - E_{\parallel 1} = 0 \) & \( E_{\parallel 2} = 0 \) & \( E_{\parallel 2} = -\mathfrak{M}_s \) \\
    \( H_{\parallel} := n \times H \) & \( H_{\parallel 2} - H_{\parallel 1} = J_s \) & \( H_{\parallel 2} - H_{\parallel 1} = 0 \) & \( H_{\parallel 2} = J_s \) & \( H_{\parallel 2} = 0 \) \\
    \( D_{\perp} := n \cdot D \) & \( D_{\perp 2} - D_{\perp 1} = \phi_{es} \) & \( D_{\perp 2} - D_{\perp 1} = 0 \) & \( D_{\perp 2} = \phi_{es} \) & \( D_{\perp 2} = 0 \) \\
    \( B_{\perp} := n \cdot B \) & \( B_{\perp 2} - B_{\perp 1} = \phi_{ms} \) & \( B_{\perp 2} - B_{\perp 1} = 0 \) & \( B_{\perp 2} = 0 \) & \( B_{\perp 2} = \phi_{ms} \) \\
    \hline
\end{tabular}
\newpage
\normalsize
\subsection{Material Considerations}
All the constitutive parameters of \eqref{eq:Constitutive} are typically time/space-varying tensors. Furthermore, they are complex-valued in order to model dissipation for time-varying fields.

Generally, we can classify materials into categories described below.

\subsubsection{Magnets} The magnetization is the net effect of the microscopic magnetic dipoles created by orbiting electrons. A large value of $\mu$ indicates a stronger magnetization.
\subsubsection{Dielectrics/Insulators}
Here, the dominant charges are on the boundary of the material creating an overall electric dipole. A large value of $\epsilon$ indicates a stronger ability to store charge, but must be weighed vs. $\sigma$ and $\omega$ also. The condition for a good dielectric is
    \begin{equation} \label{eq:dielectric}
        \frac{\sigma}{\omega \epsilon} \ll 1
    \end{equation}
\subsubsection{Conductors}
Here, there are free charges creating currents throughout the material, due to valence electrons that aren't tightly bound. The condition here is the opposite of the above:
    \begin{equation} \label{eq:conductor}
        \frac{\sigma}{\omega \epsilon} \gg 1
    \end{equation}
\subsubsection{Semiconductors}
These are roughly in between an insulator and a conductor, with the condition
    \begin{equation} \label{eq:semiconductor}
        \frac{\sigma}{\omega \epsilon} = O(1)
    \end{equation}
\newpage
\subsection{Wave Equation}
$E$ and $H$ obey equations:
\begin{subequations} \label{eq:fieldEqs}
    \begin{align}
    (\mu \epsilon \partial^2_t + \mu \sigma \partial_t - \nabla^2 ) E & + \nabla \times \mathfrak{M}_i + \mu \partial_t J_i + \frac{1}{\epsilon} \nabla \rho_e &= 0 \label{eq:electricFieldEq} \\
    (\mu \epsilon \partial^2_t + \mu \sigma \partial_t - \nabla^2 ) H & - \nabla \times J_i + \epsilon \partial_t \mathfrak{M}_i + \frac{1}{\mu} \nabla \rho_m + \sigma \mathfrak{M}_i &= 0 \label{eq:magneticFieldEq}
    \end{align}
\end{subequations}
To obtain solutions we usually look at time-harmonic solutions, and can then use Fourier series to express other forms in terms of these. The time harmonic forms of \eqref{eq:fieldEqs} are obtained by replacements $\partial_t \to \omega j$, $\partial_t^2 \to - \omega^2$
\begin{subequations} \label{eq:fieldEqsTH}
    \begin{align}
    (- \mu \epsilon \omega^2 + \mu \sigma \omega j - \nabla^2 ) E & + \nabla \times \mathfrak{M}_i + \mu \omega j J_i + \frac{1}{\epsilon} \nabla \rho_e &= 0 \label{eq:electricFieldEqTH} \\
    ( - \mu \epsilon \omega^2 + \mu \sigma \omega j - \nabla^2 ) H & - \nabla \times J_i + \epsilon \omega j \mathfrak{M}_i + \frac{1}{\mu} \nabla \rho_m + \sigma \mathfrak{M}_i &= 0 \label{eq:magneticFieldEqTH}
    \end{align}
\end{subequations}
\subsubsection{Source-Free Solutions}
The source-free ($\rho_e = \rho_m = J_i = \mathfrak{M}_i = 0$) versions of \eqref{eq:fieldEqsTH} are
\begin{subequations} \label{eq:fieldEqsSF}
    \begin{align}
    (- \mu \epsilon \omega^2 + \mu \sigma \omega j - \nabla^2 ) E & = 0 \label{eq:electricFieldEqSF} \\
    ( - \mu \epsilon \omega^2 + \mu \sigma \omega j - \nabla^2 ) H & = 0 \label{eq:magneticFieldEqSF}
    \end{align}
\end{subequations}
Solutions to \eqref{eq:fieldEqsSF} can be obtained by:
\begin{enumerate}
    \item Expressing the field in terms of coordinate functions, and
    \item Using separation of variables.
\end{enumerate}
The solutions obtained in this way are expressible in terms of complex exponentials and Bessel functions.
Note that the quantity $- \mu \epsilon \omega^2 + \mu \sigma \omega j $ can be expressed as the square of a single complex number $\gamma = \alpha + kj$. Solving for $\alpha$ and $k$, we get
\begin{equation} \label{eq:attenuationConstant}
    \alpha = \omega \sqrt{\mu \epsilon} \left( \frac{1}{2} \left( \sqrt{1 + \left( \frac{\sigma}{\omega \epsilon} \right)^2 } - 1 \right)  \right)^{\frac{1}{2}}
\end{equation}
and
\begin{equation} \label{eq:wavenumber}
    k = \omega \sqrt{\mu \epsilon} \left( \frac{1}{2} \left( \sqrt{1 + \left( \frac{\sigma}{\omega \epsilon} \right)^2 } + 1 \right)  \right)^{\frac{1}{2}}
\end{equation}
which, for lossless materials ($\sigma = 0$) reduce to
\begin{equation} \label{eq:attenuationConstantLossless}
    \alpha = 0
\end{equation}
and
\begin{equation} \label{eq:wavenumberLossless}
    k = \omega \sqrt{\mu \epsilon}
\end{equation}
\subsubsection{Transverse Modes}
Transverse modes are solutions of \eqref{eq:fieldEqsTH} whose $E$ and/or $H$ fields have no component for a given set of coordinates (it is said to be "transverse to" this set) over time for a given spatial point, e.g.
\begin{itemize}
\item $TE^y$ means that the electric field has no $y$ component,
\item $TM^z$ means that the magnetic field has no $z$ component,
\item $TEM$ means that the electric and magnetic field are both contained in a plane,
\item If equiphase planes are parallel, then it's a plane wave.
\end{itemize}
\newpage
\section{Reflection and Transmission}
\subsection{Normal Incidence}
Assuming the wave vector in the $z$ direction, the electric field is polarized in the $x$ direction, defining $E_0, \Gamma, T$ respectively by
\begin{subequations} 
    \begin{align*}
        E^i &= E_0 e^{-(\alpha_1 + jk_1) z} \mathbf{e}_x \\
        E^r &= \Gamma E_0 e^{(\alpha_1 + jk_1)z} \mathbf{e}_x \\
        E^t &= T E_0 e^{-(\alpha_2 + jk_2) z} \mathbf{e}_x \\
    \end{align*}
\end{subequations}
applying right-hand-rule and enforcing continuity of tangential components ($\Theta^i + \Theta^r = \Theta^t$ at $z=0$, where $\Theta \in {E,H}$) leads to 
\begin{subequations} \label{eq:losslessNormalReflection}
    \begin{align}
        \Gamma &= \frac{\eta_2 - \eta_1}{\eta_2 + \eta_1} \label{eq:reflectionNormal} \\
        T &= 1 + \Gamma
    \end{align}
\end{subequations}
\subsection{Oblique Incidence}
The formulae for oblique angles are simple to obtain in a similar fashion. For the electric field perpendicular to the plane of incidence we replace \eqref{eq:reflectionNormal} by
\begin{equation} \label{eq:reflectionObliquePerp}
    \Gamma_{\perp} = \frac{\eta_2 \cos \theta_i - \eta_1 \cos \theta_t }{\eta_2 \cos \theta_i + \eta_1 \cos \theta_t}
\end{equation}
and when it is polarized parallel, it instead becomes
\begin{equation} \label{eq:reflectionObliquePar}
    \Gamma_{\parallel} = \frac{\eta_2 \cos \theta_t - \eta_1 \cos \theta_i }{\eta_2 \cos \theta_t + \eta_1 \cos \theta_i}
\end{equation}
which can both be derived from the formulae for plane-wave impedances for transverse modes (the first is $TM^z$ corresponding to $\eta_p \to \frac{\eta_p}{\cos \theta_p}$, and the second is $TE^z$ corresponding to $\eta_p \to \eta_p \cos \theta_p $)
\newpage
\section{Vector Potentials}
The magnetic vector potential $A$ is defined for source-free regions (guaranteed experimentally, since there are no magnetic monopoles: $\rho_m = 0$) by
\begin{equation} \label{eq:magneticVectorPotential}
    B_A = \nabla \times A 
\end{equation}
The electric scalar potential $\phi_e$ is then defined by
\begin{equation} \label{eq:electricScalarPotential}
    E_A = - \nabla \phi_e -j \omega A 
\end{equation}
Similarly, if there are no electric charges ($\rho_e = 0$), then we can define the electric vector potential $F$ by
\begin{equation} \label{eq:electricVectorPotential}
    D_F = - \nabla \times F 
\end{equation}
and the magnetic scalar potential $\phi_m$ by
\begin{equation} \label{eq:magneticScalarPotential}
    H_F = - \nabla \phi_m -j \omega F 
\end{equation}
We can specify $\phi_e, \phi_m$ arbitrarily (doing so is called "fixing a gauge").
The Lorenz gauge is defined by
\begin{equation} \label{eq:LorenzGauge}
    \nabla \cdot \Theta + \mu \epsilon \partial_t \phi_{\theta} = 0 
\end{equation}
Applying this to both potentials leads to the Helmholtz equations:
\begin{subequations} \label{eq:Helmholtz}
    \begin{align}
        \nabla^2 A + k^2 A &= -\mu J \label{eq:HelmMag} \\
        \nabla^2 F + k^2 F &= -\epsilon \mathfrak{M} \label{eq:HelmElec}
    \end{align}
\end{subequations}
\end{document}
