\documentclass{article} % Main types described below
% article    Short documents—articles, papers, assignments. No chapters.
% report     Longer documents like theses and reports. Supports chapters.
% book       Books or manuals. Chapters start on right pages. Has parts, front/back matter.
% beamer     Slide presentations. Each frame is a slide. Highly customizable.
% letter     Writing formal letters. Simple formatting with opening/closing.
% proc       Conference proceedings. Similar to article but with different style.
\usepackage[usenames,dvipsnames]{color}  % Basic color names, extended color support
\usepackage{empheq}                      % Highlight or group equations nicely
\usepackage{enumerate}                   % Custom labels for numbered lists
\usepackage{graphicx}                    % Include and scale images
\usepackage{hyperref}                    % Clickable links and PDF bookmarks
\usepackage{tabularx}                    % Tables with adjustable-width columns
\usepackage[most]{tcolorbox}             % Versatile colored boxes for notes or theorems
\usepackage{textcomp}                    % Extra text symbols like degree and currency
\usepackage{url}                         % Properly typeset and break long URLs
\usepackage{verbatim}                    % Show text/code exactly as typed
\usepackage{wasysym}                     % Special symbols: smileys, weather, etc.
\usepackage{xcolor}                      % Advanced color control and models

\definecolor{linkborder}{RGB}{204,229,255}  % Custom color for link borders (pale blue)
\definecolor{vanish}{RGB}{224,224,224}      % Custom color for vanishing terms (pale grey)
\hypersetup{linkbordercolor=linkborder}     % Set link border color to defined pale blue
\numberwithin{equation}{section}            % Set the equation numbering to be by sections
\renewcommand{\arraystretch}{2}             % Set multiplier for default row height

\title{\LaTeX Notes}
\author{Andrew Whelan}
\date{\today}
\begin{document}
    \maketitle
    \thispagestyle{empty}   % No numbering on this page
    \setcounter{page}{0}    % Set the undisplayed page number to 0
    \tableofcontents        % For each unnumbered section add something like
                            % `\addcontentsline{toc}{section}{<section_name>}`
    \newpage                
    \section{Notes from Overleaf \href{https://www.overleaf.com/learn}{*}}

        Using \verb|\begin{description}| allows you to set up some examples like in the following:
        \begin{description}
            \item[Example 1] This is the first example
            \item[Example 2] ...and this is the second. With some much longer text to show the overall formatting. Oh dear, isn't this rather wordy!
        \end{description}
        \begin{description}
            \item[Quoting Commands] Using \verb|\verb| will allow you to quote things better.
        \end{description}
\end{document}
