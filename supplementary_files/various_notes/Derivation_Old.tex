\section{P.E.C Field Derivation}
    \begin{enumerate}
        \item   Starting from Maxwell's equations \eqref{eq:Maxwell} and constitutive equations \eqref{eq:Constitutive}, transform to frequency domain to simplify curl equations:
                \begin{subequations}\label{eq:MaxwellFD}
                    \begin{align}
                        \nabla \times H &= J + j \omega D \label{eq:AmpereMaxwellFD} \\
                        \nabla \times E &= - j \omega B \label{eq:MaxwellFaradayFD}
                    \end{align}
                \end{subequations}
        \item   Introduce magnetic vector potential $A$ and electric potential $\phi$ defined by:
                \begin{subequations}\label{eq:Potentials}
                    \begin{align}
                        B &= \nabla \times A \label{eq:MagneticPotential} \\
                        E &= - \nabla \phi - j \omega A \label{eq:ElectricPotential}
                    \end{align}
                \end{subequations}
                and fix Lorenz gauge:
                \begin{equation} \label{eq:LorenzGaugeCondition}
                    \nabla \cdot A = - \frac{j \omega}{c^2} \phi
                \end{equation}
        \item   Apply the above with some vector identities to reduce Ampere-Maxwell equation \eqref{eq:AmpereMaxwellFD} to Helmholtz equation:
                \begin{enumerate}
                    \item Substitute the constitutive conditions \eqref{eq:Constitutive} and potentials \eqref{eq:Potentials} respectively to \eqref{eq:AmpereMaxwellFD}:
                        \begin{align*}
                            & \frac{1}{\mu} \nabla \times B = J + j \omega \epsilon E \\
                            \implies & \frac{1}{\mu} \nabla \times \nabla \times A = J - j \omega \epsilon \nabla \phi - \omega^2 \epsilon A
                        \end{align*}
                    \item Use the vector calculus identity $\nabla \times \nabla \times A = \nabla(\nabla \cdot A) - \nabla^2 A$, so
                        \begin{align*}
                            \frac{1}{\mu} \left( \nabla(\nabla \cdot A) - \nabla^2 A \right) &= J - j \omega \epsilon \nabla \phi - \omega^2 \epsilon A
                        \end{align*}
                    \item Apply the Lorenz gauge condition \eqref{eq:LorenzGaugeCondition}:
                        \begin{align*}
                            & -\frac{1}{\mu} ( \frac{j \omega}{c^2} \nabla \phi + \nabla^2 A ) = J - j \omega \epsilon \nabla \phi - \omega^2 \epsilon A \\
                            \implies & - \frac{1}{\mu} \nabla^2 A = J - \omega^2 \epsilon A \\
                            \implies & \nabla^2 A + k^2 A = -\mu J.
                        \end{align*}
                \end{enumerate}
        \item   Solve for vector potential using Green functions to get
                \begin{equation} \label{eq:GreenSol1}
                    A = \mu \int J(r) G(r, r') dr'
                \end{equation}
        \item   Plug into electric field formula \eqref{eq:ElectricPotential} in terms of potentials:
                \begin{align*}
                    & A = \frac{j}{\omega} ( \nabla \phi + E )
                \end{align*}
                Use the far field assumption on the scalar potential $\nabla \phi \to 0$ - this can be worked out to die off quickly asymptotically. Finally we arrive at
                \begin{align*}
                    & E \approx - \omega \mu j \int J(r) G(r, r') dr' \\
                    \implies & E \approx - \eta k j \int J(r) G(r, r') dr'
                \end{align*}

    \end{enumerate}


