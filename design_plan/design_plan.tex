\documentclass{article} % Main types described below
% article    Short documents—articles, papers, assignments. No chapters.
% report     Longer documents like theses and reports. Supports chapters.
% book       Books or manuals. Chapters start on right pages. Has parts, front/back matter.
% beamer     Slide presentations. Each frame is a slide. Highly customizable.
% letter     Writing formal letters. Simple formatting with opening/closing.
% proc       Conference proceedings. Similar to article but with different style.

\usepackage[usenames,dvipsnames]{color}  % Basic color names, extended color support
\usepackage{empheq}                      % Highlight or group equations nicely
\usepackage{enumerate}                   % Custom labels for numbered lists
\usepackage{enumitem}
\usepackage{graphicx}                    % Include and scale images
\usepackage{hyperref}                    % Clickable links and PDF bookmarks
\usepackage{tabularx}                    % Tables with adjustable-width columns
\usepackage[most]{tcolorbox}             % Versatile colored boxes for notes or theorems
\usepackage{textcomp}                    % Extra text symbols like degree and currency
\usepackage{titling}
\usepackage{url}                         % Properly typeset and break long URLs
\usepackage{verbatim}                    % Show text/code exactly as typed
\usepackage{wasysym}                     % Special symbols: smileys, weather, etc.
\usepackage{xcolor}                      % Advanced color control and models

\definecolor{linkborder}{RGB}{204,229,255}  % Custom color for link borders (pale blue)
\definecolor{vanish}{RGB}{224,224,224}      % Custom color for vanishing terms (pale grey)
\hypersetup{linkbordercolor=linkborder}     % Set link border color to defined pale blue
\numberwithin{equation}{section}            % Set the equation numbering to be by sections
\renewcommand{\arraystretch}{2}             % Set multiplier for default row height
\setlist[description]{labelindent=1em, leftmargin=1em, labelwidth=2em, labelsep=3em}

\title{Modelling of Diffuse Scattering Effects for Outdoor Ray-Tracing}
\author{Andrew Whelan}
\date{}

%\subtitle{Project Design Plan}
\begin{document}
    %\maketitle
    \begin{titlepage}
        \centering
        \vspace*{\fill}
        {\Large\bfseries \thetitle \par}
        \vspace{3em}
        {\large\bfseries Project Design Plan \par}
        \vspace{3em}
        {\large\theauthor \par}
        % \vspace{1em}
        % {\large Subtitle Here}  % Add if desired

        \vspace*{\fill}
        \thispagestyle{empty}   % No numbering on this page
        \setcounter{page}{0}    % Set the undisplayed page number to 0
    \end{titlepage} 
    \newpage                
    \paragraph*{Research Question:} In what ways can current heuristic 
    models of scattering be critically assessed and developed to improve their
    predictive reliability?
    \paragraph*{Project Scope:} % Bullet point list of approaches, technologies, research 
%                                 areas/technologies/considerations addressed or not.
        \begin{itemize}
            \item Formulation of the Effective Roughness Approach, with known
                  limitations
            \item Basic plane-wave setup and comparison of Effective Roughness model 
                  vs.:
            \begin{itemize}
                \item Geometrical Optics 
                \item Physical Optics 
                \item Method of Moments
            \end{itemize}
            \item Variation of setup parameters and its effect on Effective-Roughness
                  predictability, including:
            \begin{description}
                \item[$\lambda$] wavelength
                \item[$\Delta w$] numerical-integration step-size
            \end{description}
        \end{itemize}

\end{document}
