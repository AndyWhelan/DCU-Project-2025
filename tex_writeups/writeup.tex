\documentclass{article}

\usepackage{amsfonts}
\usepackage{graphicx}
\usepackage{mathrsfs}
\usepackage{mathtools}
\usepackage{subcaption}

\DeclareMathOperator{\sign}{sgn}
\DeclarePairedDelimiter\abs{\lvert}{\rvert}
\DeclarePairedDelimiter\norm{\lVert}{\rVert}
\begin{document}
% newcommand defs {{{
\newcommand{\ampRep}{\vec{A}}
\newcommand{\ampCoef}{\ampRep_m}
\newcommand{\freq}{\omega}
\newcommand{\cPoynt}{\mathbf{S}}
\newcommand{\cond}{\sigma}
\newcommand{\curv}{\kappa}
\newcommand{\defeq}{\overset{\mathclap{\text{def}} }{=}}
\newcommand{\eFreq}{\mathbf{E}}
\newcommand{\eFreqMom}{\tilde{\eFreq}}
\newcommand{\eikonal}{\mathcal{L}}
\newcommand{\complexNums}{\mathbb{C}}
\newcommand{\dispRel}{k}
\newcommand{\eTime}{\mathscr{E}}
\newcommand{\eqBy}[1]{\overset{\mathclap{\eqref{#1} }}{=}}
\newcommand{\fourier}{\mathcal{F}}
\newcommand{\fresnel}{\Gamma}
\newcommand{\hFreq}{\mathbf{H}}
\newcommand{\hTime}{\mathscr{H}}
\newcommand{\im}{j}
\newcommand{\impedance}{\eta}
\newcommand{\incField}{\eFreq_i}
\newcommand{\initField}{\eFreq_0}
\newcommand{\laplacian}{\nabla^2}
\newcommand{\permeab}{\mu}
\newcommand{\permitt}{\epsilon}
\newcommand{\pos}{\vec{r}}
\newcommand{\posMag}{r}
\newcommand{\incPt}{\pos_i}
\newcommand{\focalPt}{\vec{f}_r}
\newcommand{\posNorm}{\hat{r}}
\newcommand{\incPtNorm}{\posNorm_i}
\newcommand{\poyntAvg}{\langle \poynt \rangle}
\newcommand{\normV}{\vec{n}}
\newcommand{\normVNorm}{\hat{n}}
\newcommand{\poynt}{\mathscr{S}}
\newcommand{\reals}{\mathbb{R}}
\newcommand{\surf}{\Sigma}
\newcommand{\posSurf}{\pos_{\surf}}
\newcommand{\tx}{\pos_t}
\newcommand{\txNorm}{\posNorm_t}
\newcommand{\wavenum}{\dispRel( \freq )}
\newcommand{\wavevec}{\mathbf{k}}
\newcommand{\wavevecNorm}{\hat{\wavevec}}
\newcommand{\reflNorm}{\wavevecNorm_r}
\newcommand{\reflPt}{\pos_r}
\newcommand{\txPower}{P_0}
\newcommand{\unitTang}{\hat{t}}
\newcommand{\unitNorm}{\hat{n}}
\newcommand{\kra}{\wavevec_i + 2 \dispRel \cos \theta_i \unitNorm(0)}
\newcommand{\krb}{(-2 \dispRel \curv_{\surf}(0) \cos \theta_i ) \unitTang(0) + (2
                  \dispRel \curv_{\surf}(0) \sin \theta _i ) \unitNorm(0) }
\newcommand{\krca}{\left( - 2 \dispRel \curv_{\surf}'(0) \cos \theta_i - 4 \dispRel
                   \curv_{\surf}^2(0) \sin \theta_i \right) \unitTang(0)}
\newcommand{\krcb}{\left( 2 \dispRel \curv_{\surf}'(0) \sin \theta_i  - 4 \dispRel
                   \curv_{\surf}^2(0) \cos \theta_i \right) \unitNorm(0)}
% }}}
\numberwithin{equation}{section}
\section{Geometrical Optics}
Geometrical Optics can be derived as a large $\freq$ limit of wave optics. We begin
with the time-independent Helmholtz equation for a general lossless, homogeneous
medium: 
\begin{equation}
   ( \laplacian + \wavenum^2 ) \eFreq = 0
   \label{eq:helmholtz}
\end{equation}
where $\dispRel : \reals \to \reals$ is the dispersion relation, $\wavenum$
is the ``wavenumber'', and $\eFreq : \reals^{3} \times \reals \to \complexNums^3$ is
the electric field in the frequency domain.

Assume a plane-wave Fourier decomposition (ignore the normalization factor as it can
be factored out of \eqref{eq:helmholtz}):
\begin{equation}
   \eFreq( \pos, \freq ) = \iiint_{\reals^3} e^{\im \wavevec \cdot \pos} \ \eFreqMom(
      \wavevec, \freq ) d^3 \wavevec
   \label{eq:planeWaveDecomposition}
\end{equation}
Then, substituting \eqref{eq:planeWaveDecomposition} into \eqref{eq:helmholtz} we
ultimately get
\begin{equation}
   \norm{\wavevec}^2 - \wavenum^2 = 0
   \label{eq:waveCond}
\end{equation}
\subsection{Rays as Limit of Wave Optics}
From Maxwell's equations, the dispersion relation in a lossless medium is $\wavenum =
\freq \sqrt{\permeab \permitt}$, so that in the high frequency limit, the integral in
\eqref{eq:planeWaveDecomposition} oscillates rapidly and is dominated by
contributions where $\wavevec \cdot \pos$ is stationary. 

Applying a stationary phase variational principle to find these stationary points,
subject to the constraint from \eqref{eq:waveCond}, we get
\begin{align}
   \delta(\wavevec \cdot \pos) &= 0 \nonumber \\
   \implies (\delta \wavevec) \cdot \pos &= 0, \quad \forall \delta \wavevec \ \text{
      satisfying \eqref{eq:waveCond}}
\label{eq:stationaryCondition}
\end{align}
Taking the variation of \eqref{eq:waveCond} gives:
\begin{align}
   \delta (\norm{\wavevec}^2 - \wavenum^2) = \delta(\wavevec \cdot \wavevec) &= 0
      \nonumber \\
   \implies 2(\wavevec \cdot \delta \wavevec) &= 0 \nonumber \\
   \implies \wavevec \cdot \delta \wavevec&=0
\label{eq:constraintCondition}
\end{align}
Since the variation is constrained by \eqref{eq:waveCond}, $\delta \wavevec$ spans the
tangent plane to the sphere defined by \eqref{eq:waveCond}. For
\eqref{eq:stationaryCondition} and \eqref{eq:constraintCondition} to hold for all
variations, we must therefore have $\pos$ normal to the tangent plane spanned by
$\delta \wavevec$, and so we must have
\begin{equation}
   \pos \parallel \wavevec_s, \quad \text{for stationary } \wavevec_s
\label{eq:rayCondition}
\end{equation}
This result proves that the field at an observation point $\pos$ is dominated by the
plane wave component whose wavevector $\wavevec_s$ points along the line connecting
the origin to $\pos$. This is the mathematical emergence of a ray and the principle
of straight-line propagation.
\subsection{Ray Evolution in 2D}
Condition \eqref{eq:waveCond} combined with already being in $\reals^2$, means we can
restrict ourselves to a 1d integral. Take a parametrization $\wavevec(s)$, where $s$
is the arc-length, and expand the phase $\phi(s)$ in a Taylor-series:
\begin{align}
   \phi(s) & \defeq \pos \cdot \wavevec (s) \nonumber \\
   &= \pos \cdot \left( \wavevec( 0 ) + \underbrace{\wavevec'(0)}_{\wavevec'(0) \perp
      \pos} s + \frac{\wavevec''(0)}{2} s^2 + \underbrace{\ldots}_{\text{small}}
      \right) \nonumber \\
   & \approx \pos \cdot \left( \wavevec(0) + \frac{\wavevec ''(0)}{2} s^2 \right)
   \label{eq:taylorPhi}
\end{align}
Now, use the WKB method of assuming the amplitude is slowly varying compared to
$\phi$, so that $\eFreqMom(s) \approx \eFreqMom(0)$. Then
\eqref{eq:planeWaveDecomposition} becomes
\begin{align}
   \eFreq & \approx \eFreqMom (0) e^{\im \pos \cdot \wavevec(0)} \int_{\reals}
      e^{\im \pos \cdot \wavevec'' (0) \frac{s^2}{2}} ds \nonumber \\
   &= \eFreqMom (0) e^{\im \pos \cdot \wavevec(0)} \sqrt{\frac{2\pi}{-j
      \pos \cdot \wavevec''(0)}} \nonumber \\
   &= \sqrt{\frac{-2 \pi \dispRel \im }{ \posMag \ \posNorm \cdot
      \wavevecNorm( 0 ) }} \ \eFreqMom (0) e^{\im \left( \left( \dispRel
      \posMag \right) \ \posNorm \cdot \wavevecNorm(0) \right)} \nonumber \\
   \implies \eFreq & \approx \sqrt{\frac{2 \pi \dispRel}{\posMag}} \ \eFreqMom(0) \
      e^{\pm \im \left( \dispRel \posMag \ - \ \underbrace{\pi/4}_{\text{Guoy shift}}
      \right) }
\end{align}
The term $\eFreqMom(0)$ represents the source's radiation strength in the direction
of the observer. By conserving power, it can be shown that this term is related to
the total power radiated by the source, $\txPower$:
\begin{equation}
   |\eFreqMom(0)| = \sqrt{\frac{\impedance \txPower F(\wavevecNorm_s)}{2\pi^2 \dispRel}}
\end{equation}
where $\impedance$ is the impedance of the medium and $F(\wavevecNorm_s)$ is the dimensionless radiation pattern of the source (equal to 1 for an isotropic radiator). Substituting this back gives the final, physically interpretable result for the electric field:
\begin{equation}
   \eFreq(\pos) \approx \sqrt{\frac{\impedance \txPower F(\wavevecNorm_s)}{\pi
     \posMag}} \ e^{ \pm \im \left( \dispRel \posMag - \pi/4 \right)}
   \label{eq:rayEvolution2d}
\end{equation}
\subsection{Reflection and Caustics}

Due to linearity, planar reflection can always be solved by decomposing fields into
components parallel and perpendicular to the plane of incidence, and then applying
the usual Fresnel linear map $\fresnel : \complexNums^2 \to \complexNums^2$.

Curved surfaces are a bit more complicated. For a 2d surface $\surf : \reals \to
\reals^2$ with curvature $\curv_{\surf}$, the $\posMag_i$ in
\eqref{eq:rayEvolution2d} will change to $\posMag_r$ after hitting $\surf$.
Considering high frequency radiation coming from $\Sigma$, the phase to vary is
\begin{equation}
   \phi_{r}(s) = ( \pos_r - \posSurf(s) ) \cdot \wavevec_r(s)
   \label{eq:phaseFromSurface}
\end{equation}
Then, expanding analagously to \eqref{eq:quadraticPhaseApprox}:
\begin{align}
   \posSurf(s) & \approx \posSurf(0) + s \posSurf \ '(0) + \frac{s^2}{2} \posSurf \
      ''(0) \nonumber \\
   &= \posSurf(0) + s \unitTang( 0 ) + \frac{s^2}{2} \curv_{\surf} (0) \unitNorm (0 )
   \label{eq:surfaceApprox}
\end{align}
where $\unitTang$ and $\unitNorm$ are the unit tangent and normal to $\surf$
respectively, and we have used the arc-length parametrization and applied
Frenet-Serret formulae. We also have 
\begin{align}
   \wavevec_r(s) & \approx \wavevec_r(0) + s \wavevec_r '(0) + \frac{s^2}{2}
   \wavevec_r ''(0)
\end{align}
and by the reflection law identity (using the tangent plane approximation)
\begin{align}
   \wavevec_r(s) = \wavevec_i - 2 (\unitNorm(s) \cdot \wavevec_i) \unitNorm(s)
      \label{eq:reflectionLaw}
\end{align}
From \eqref{eq:reflectionLaw} and Frenet-Serret, we get
\begin{subequations}
\begin{align}
   \wavevec_r(0) &= \wavevec_i - 2 (\unitNorm(0) \cdot \wavevec_i) \unitNorm(0) \\
   \wavevec_r'(0) &= 2 \curv_{\surf}(0) ( \unitNorm(0) \cdot \wavevec_i )
      \unitTang(0) + 2 \curv_{\surf}(0) (\unitTang(0) \cdot \wavevec_i ) \unitNorm(0) \\
   \wavevec_r''(0) &= \left( 2 \curv_{\surf}'(0) ( \unitNorm(0) \cdot \wavevec_i ) -
      4 \curv_{\surf}^2(0) (\unitTang(0) \cdot \wavevec_i) \right) \unitTang(0) \ +
      \nonumber \\
   & \quad \ \left( 2 \curv_{\surf}'(0) ( \unitTang(0) \cdot \wavevec_i ) + 4
      \curv_{\surf}^2(0) (\unitNorm(0) \cdot \wavevec_i) \right) \unitNorm(0)
\end{align}
\end{subequations}
These can be simplified further, since $\unitNorm(0) \cdot \wavevec_i = -\dispRel
\cos \theta_i$, and $\unitTang(0) \cdot \wavevec_i = \dispRel \sin \theta_i$:
\begin{subequations}
\begin{align}
   \wavevec_r(0) &= \wavevec_i + 2 \dispRel \cos \theta_i \unitNorm(0) \\
   \wavevec_r'(0) &= (-2 \dispRel \curv_{\surf}(0) \cos \theta_i )
      \unitTang(0) + (2 \dispRel \curv_{\surf}(0) \sin \theta_i ) \unitNorm(0) \\
   \wavevec_r''(0) &= \left( - 2 \dispRel \curv_{\surf}'(0) \cos \theta_i -
      4 \dispRel \curv_{\surf}^2(0) \sin \theta_i \right) \unitTang(0) \ +
      \nonumber \\
   & \quad \ \left( 2 \dispRel \curv_{\surf}'(0) \sin \theta_i  - 4
      \dispRel \curv_{\surf}^2(0) \cos \theta_i \right) \unitNorm(0)
\end{align}
\end{subequations}
Plugging these into \eqref{eq:phaseFromSurface}, and neglecting terms larger than
$s^2$, we get
\begin{align}
   \phi_r(s) & \approx (\pos_r - \pos_{\surf}(0)) \cdot ( \kra ) \nonumber \\
   &+ s \Large( ( \pos_r - \pos_{\surf}(0) ) \cdot \krb ) ) \nonumber \\
   &- s \unitTang(0) \cdot ( \kra ) \nonumber \\
   &+ \frac{s^2}{2} ( \pos_r - \pos_{\surf}(0) ) \cdot ( \krca ) \nonumber \\
   &+ \frac{s^2}{2} ( \pos_r - \pos_{\surf}(0) ) \cdot ( \krcb ) \nonumber \\
   &- \frac{s^2}{2}  \curv_{\surf}(0) \unitNorm(0) \cdot ( \kra ) \\
   &- s^2 \unitTang(0) \cdot( \krb ) 
   \label{eq:reflectionPhase}
\end{align}
Using the following conventions 
\begin{subequations}
\begin{align}
   r_t & \defeq (\pos_r - \pos_{\Sigma}(0)) \cdot \unitTang(0) \\ 
   r_n & \defeq (\pos_r - \pos_{\Sigma}(0)) \cdot \unitNorm(0)
\end{align}
\end{subequations}
we can then simplify the dot products in terms of angle of incidence, ending up with
a quadratic phase $as^2 + bs + c$ where 
\begin{subequations}
\begin{align}
   a &= \begin{aligned}[t]
      \frac{\dispRel}{2} \biggl[ & \left(-2 \curv_{\surf}'(0) \cos\theta_i - 4
      \curv_{\surf}^2(0) \sin\theta_i \right) r_t \\
      & + \left(2 \curv_{\surf}'(0)
      \sin\theta_i - 4 \curv_{\surf}^2(0) \cos\theta_i \right) r_n \\
      & + 3 \curv_{\surf}(0)\cos\theta_i \biggr]
   \end{aligned} \\
   b &= 2 \dispRel \curv_{\surf}(0) (r_n \sin\theta_i - r_t \cos\theta_i) - \dispRel
   \sin\theta_i \\
   c &= \dispRel (r_t \sin\theta_i + r_n \cos\theta_i)
\end{align}
\end{subequations}
So, under the WKB approximation as before, \eqref{eq:planeWaveDecomposition} reduces
to
\begin{equation}
   \eFreq(\pos) = \eFreqMom(0) \sqrt{\frac{\pi}{\lvert a \rvert}} \ e^{ \pm j \left(c
      - \frac{b^2}{4a} + \frac{\pi}{4} \right)}
   \label{eq:unexpandedReflGO}
\end{equation}
The stationary phase assumption tells us $b=0$, but it's left in here because
Assuming a setup with a fixed transmitter $\tx$ and receiver $\reflPt$, we want to
use the stationary phase principle to find the stationary points $\incPt$ along a
planar surface $P$. To do this, the Fourier plane wave representation We can choose a Fourier representation The phase $\wavevec \cdot \pos$ of
\eqref{eq:planeWaveDecomposition} can be written as 
\begin{equation}
   \wavevec \cdot \pos(s) = \wavevec \cdot \left( (\incPt(s) - \tx ) + ( \reflPt -
      \incPt(s) ) \right)
   \label{eq:phaseReflection}
\end{equation}
\subsection{Scattering from rough surfaces}
When rays are transmitted from a known point source $\tx$, we will assume the outgoing
spreading to be symmetric concentric circles. The incident field is then given by 
\begin{equation}
   \incField (\incPt) = \sqrt{\frac{P_0 \impedance}{2 \pi \norm{\incPt - \tx}}} e^{- \im k
   \norm{\incPt - \tx}} \ \wavevecNorm_i
   \label{eq:incidentGOField}
\end{equation}
In the case of a rough surface $\surf:
\reals \to \reals^2$, incident rays are assumed to reflect in a cylindrical wave
whose curvature $\curv_r$ (using the convention that $\curv < 0$ corresponds to
diverging waves) is related to 
\begin{itemize}
   \item the curvature of the wavefront $\curv_i = -\frac{1}{\norm{\incPt - \tx}}$,
      and 
   \item the curvature of the surface at that point $\curv_{\surf}$
\end{itemize}
according to 
\begin{align}
   \curv_r &= - 2 \curv_{\surf} \cos \theta_i - \frac{1}{\norm{\incPt - \tx}} \nonumber \\
   &= - 2 \curv_{\surf} | \wavevecNorm_i \cdot \normVNorm | - \frac{1}{|\incPt - \tx|}
   \label{eq:outgoingCurvature}
\end{align}
This wave will have a corresponding apparent source whose focus will differ, but
whose spreading factor can be calculated as 
\begin{align}
   \text{spread}_{\surf} &= \frac{1}{\sqrt{1 + |\reflPt - \incPt| \ \curv_r }}
      \nonumber \\
   &= \frac{1}{\sqrt{1 - |\reflPt - \incPt| \ \left( 2
      \curv_{\surf} | \wavevecNorm_i \cdot \normVNorm | + \frac{1}{|\incPt -
      \tx|}\right) }}
      \label{eq:spreadingFactorObstruction}   
\end{align}
This poses a predictive problem for the theory in the case of focusing rays, since we
can always choose an $\reflPt$ such that the denominator of
\eqref{eq:spreadingFactorObstruction} is $\leq 0$. This is the classical problem of
\emph{caustics}, places where rays converge tightly.
\subsubsection{Wave Theory Correction at a Caustic}
To resolve the singularity, we must return to the underlying wave theory. The field reflected from the surface $\surf$ to an observation point $\reflPt$ can be expressed by the Kirchhoff-Huygens diffraction integral. For a 2D problem, this takes the form:
\begin{equation}
    \eFreq(\reflPt) = \int_{\surf} A(s) e^{\im \Phi(s)} ds
    \label{eq:kirchhoffIntegral}
\end{equation}
where $s$ is the arc length along the surface $\surf$, $A(s)$ is a slowly varying amplitude function, and $\Phi(s)$ is the phase function. The phase is determined by the total optical path length from the source $\tx$ to the observer $\reflPt$ via the point $\incPt(s)$ on the surface:
\begin{equation}
    \Phi(s) = \dispRel \left( |\incPt(s) - \tx| + |\reflPt - \incPt(s)| \right)
\end{equation}
The method of stationary phase states that for large $\dispRel$, the integral is dominated by points $s_0$ where the phase is stationary, i.e., $\frac{d\Phi}{ds} \big|_{s_0} = 0$. This condition recovers Snell's law of reflection at the specular point $s_0$.

The GO amplitude, and hence the spreading factor in \eqref{eq:spreadingFactorObstruction}, is derived from the second-order term in the Taylor expansion of the phase:
\begin{equation}
    \Phi(s) \approx \Phi(s_0) + \frac{1}{2} \Phi''(s_0) (s-s_0)^2
    \label{eq:quadraticPhaseApprox}
\end{equation}
The resulting field amplitude is proportional to $1/\sqrt{|\Phi''(s_0)|}$. The
spreading factor of \eqref{eq:spreadingFactorObstruction} is precisely this term:
\begin{equation}
    |\text{spread}_{\surf}| \propto \frac{1}{\sqrt{|\Phi''(s_0)|}}
\end{equation}
The caustic is therefore the locus of points $\reflPt$ for which the corresponding specular point $s_0$ satisfies:
\begin{equation}
    \Phi''(s_0) = 0
    \label{eq:causticCondition}
\end{equation}
This is the mathematical root of the GO failure. To find the field at or near the caustic, we must retain the next non-vanishing term in the Taylor expansion of the phase around $s_0$:
\begin{equation}
    \Phi(s) \approx \Phi(s_0) + \Phi'(s_0) (s-s_0) + \frac{1}{2}\underbrace{\Phi''(s_0)}_{=0} (s-s_0)^2 + \frac{1}{6} \Phi'''(s_0) (s-s_0)^3
\end{equation}
Here, we are at a point $\reflPt$ *near* the caustic, so the stationary phase point $s_0$ on the surface is slightly perturbed, meaning $\Phi'(s_0)$ is a small, non-zero value. It represents the offset of the observation point from the geometric caustic ray. We define:
\begin{align}
    \alpha &\defeq \Phi'(s_0) \\
    \beta &\defeq \frac{1}{6} \Phi'''(s_0)
\end{align}
The phase expansion simplifies to a cubic polynomial. Substituting this into the integral \eqref{eq:kirchhoffIntegral} gives:
\begin{equation}
    \eFreq(\reflPt) \approx A(s_0) e^{\im \Phi(s_0)} \int_{-\infty}^{\infty} e^{\im (\alpha \xi + \beta \xi^3)} d\xi
    \label{eq:protoAiry2D}
\end{equation}
where we have set $\xi = s-s_0$ and extended the integration limits to infinity, as the contribution is negligible far from $s_0$. The integral in \eqref{eq:protoAiry2D} is the canonical representation of the **Airy function**, $Ai(z)$. With a final change of variables, $t = (3\beta)^{1/3} \xi$, the integral becomes:
\begin{equation}
    \int_{-\infty}^{\infty} e^{\im (\alpha \xi + \beta \xi^3)} d\xi = \frac{1}{(3\beta)^{1/3}} \int_{-\infty}^{\infty} e^{\im \left( \frac{\alpha}{(3\beta)^{1/3}} t + \frac{1}{3} t^3 \right)} dt \propto \text{Ai}\left(\frac{\alpha}{(3\beta)^{1/3}}\right)
\end{equation}
Thus, the electric field near the caustic is no longer infinite but is described by the Airy function:
\begin{equation}
    \eFreq(\reflPt) \propto \text{Ai}\left( \frac{\Phi'(s_0)}{( \frac{1}{2} \Phi'''(s_0) )^{1/3}} \right)
\end{equation}
This solution is finite at the caustic (where $\alpha=\Phi'=0$), oscillates on the illuminated side (where multiple rays exist), and decays exponentially into the shadow region, providing a complete and physically accurate description that bridges the gap where geometrical optics fails.
\end{document}
