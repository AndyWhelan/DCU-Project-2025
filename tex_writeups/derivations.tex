% LaTeX Boilerplate
% LaTeX Boilerplate:    Packages {{{
\documentclass{article}
\usepackage{algorithmic}
\usepackage{algorithm}
\usepackage{amsmath,amsfonts}
\usepackage{array}
\usepackage{arydshln} % for horizontal dashed lines in table
\usepackage{balance}
\usepackage{cite}
\usepackage{gensymb} % for degrees symbol
\usepackage{graphicx}
\usepackage{stfloats}
\usepackage[caption=false,font=normalsize,labelfont=sf,textfont=sf]{subfig}
\usepackage{tabularx}
\usepackage{textcomp}
\usepackage{tikz}
\usetikzlibrary{calc}
\usetikzlibrary{math}
\usepackage{url}
\usepackage{verbatim}
\usepackage{xcolor}
\setlength{\parindent}{0pt}
\setlength{\parskip}{1em}
%}}}
\begin{document}
% ---Models---
% Models:               ER1 {{{
\section{ER Model} 
\subsection*{Assumptions}
\begin{itemize}
      \item \textbf{$E$-field decomposition}:
      \begin{equation}
         E = E_i + E_r + E_s + E_t, \ \text{where}
         \label{eq:eDecomp}
      \end{equation}
      \begin{itemize}
         \item $E$ is the total phasor (spatial, or time-averaged) component of the
            electric field,
         \item $E_i$ is the incident field, 
         \item $E_r$ is the specularly reflected field, 
         \item $E_s$ is the diffusely scattered field, and 
         \item $E_t$ is the transmitted field.
      \end{itemize}

   \item \textbf{$E_s$-incoherence in Rx-region}: The various contributions to $E_s$
      arriving at the Rx have random, uncorrelated phases. Consequently, the total
      squared-magnitude $|E_s|^2$ of the scattered field satisfies:
      \begin{equation}
         |E_{s}|^2 = \int_W d(|E_s|^2), \ \text{where}
         \label{eq:incoherence}
      \end{equation}
      \begin{itemize}
         \item $d(|E_s|^2)$ is the squared-field contribution from 
         \item $dW$, the infinitesimal scattering element on 
         \item $W$, the wall surface.

      \end{itemize}
   \item \textbf{Constant Local Power Balance (diffuse scattering coefficient $S$)}: A
      constant fraction $S^2$ of the power incident on a patch $dW$ is converted into
      diffusely scattered power:
      \begin{equation}
         S^2 := \frac{dP_s}{dP_i}, \ \text{where}
         \label{eq:sDef}
      \end{equation}
   \begin{itemize}
      \item $dP_i$ is the power incident on $dW$, and 
      \item $dP_s$ is the total power diffusely scattered from $dW$ into the
         surrounding hemisphere.
   \end{itemize}

      \item \textbf{Lambertian Scattering Pattern}:
      \begin{equation}
         \frac{d(dP_s)}{d\Omega_s} = K \cos \theta_s, \ \text{where}
         \label{eq:lambertianLawGeneric}
      \end{equation}
      \begin{itemize}
         \item $d \Omega_s$ is the infinitesimal solid angle of the scattered
            ray-tube, and
         \item $K$ is a normalization constant, found by integrating over the entire
            forward scattering space (a hemisphere in 3D, a semicircle in 2D) and
            setting the result equal to the total scattered power, $dP_s$.
      \end{itemize}
\end{itemize}
\subsection*{2D-Setup}
\subsubsection*{$d P_{Rx}$ - power through Rx aperture}
From the scattering element's ($dW$) POV, we have:
\begin{equation}
   dP_{Rx} \approx \left( \frac{d (dP_s) }{d \theta_s} \right) d \theta_{Rx} \label{eq:powerThroughRxApScat}
\end{equation}
Integrating \eqref{eq:lambertianLawGeneric} over $-\frac{\pi}{2} < \theta_{s}
\leq \frac{\pi}{2}$, then plugging into \eqref{eq:powerThroughRxApScat} we get 
\begin{equation}
   dP_{Rx} = \frac{dP_{s}}{2} \cos \theta_s d \theta_{Rx} = \frac{dP_s}{2r_s}
   \cos \theta_s d l_{Rx}
   \label{eq:powerPerAngleScat}
\end{equation}
From the Rx's POV, we have
\begin{equation}
   dP_{Rx} = \frac{d(|E_s|^2)}{2 \eta_0} dl_{Rx}
   \label{eq:powerPerAngleRx}
\end{equation}
Combining \eqref{eq:sDef}, \eqref{eq:powerPerAngleScat} and \eqref{eq:powerPerAngleRx} we get 
\begin{equation}
   d(|E_s|^2) = \frac{\eta_0 S^2 \cos \theta_s}{r_s} dP_i
   \label{eq:diffContrFrompowerPerAngle}
\end{equation}
\subsubsection*{$dP_i$ - power incident on $dW$}
From the incoming waves POV, we have 
\begin{equation}
   dP_i = \frac{|E_i|^2}{2 \eta_0} \cos \theta_i dx.
   \label{eq:incomingWavePower}
\end{equation}
Combining \eqref{eq:incomingWavePower} with \eqref{eq:diffContrFrompowerPerAngle}, we get
\begin{equation}
   d(|E_s|^2) = \frac{S^2 \cos \theta_i \cos \theta_s}{2r_s} |E_i|^2 dx
   \label{eq:fieldPre}
\end{equation}
\subsubsection*{$|E_i|$ as a function of $P_i$ and $r_i$}
A far-field cylindrical incoming wave allows us to assume 
\begin{equation}
   |E_i|^2 = \frac{A}{r_i}
   \label{eq:farField2d}
\end{equation}
\begin{align*}
   \implies P_i &= \int_0^{2 \pi} \frac{|E_i|^2}{2 \eta_0} r_i d \theta \\
   &= \frac{A}{2 \eta_0} \int_0^{2 \pi} d \theta \\ 
   &= \frac{A \pi}{\eta_0} \\
   \implies A &= \frac{\eta_0 P_i}{\pi}
\end{align*}
\begin{equation}
   \implies |E_i|^2 = \frac{\eta_o P_i}{\pi} \frac{1}{r_i}
   \label{eq:2dIntensityExpr}
\end{equation}
Then, combining \eqref{eq:2dIntensityExpr} with \eqref{eq:fieldPre}, we get
\begin{equation}
   d(|E_s|^2) = \frac{S^2 \eta_0 P_i \cos \theta_i \cos \theta_s }{2 \pi r_s r_i} \ dx
   \label{eq:fieldMagEq1}
\end{equation}

\subsubsection*{Coordinate Transformations}
Our setup consists of a transmitter at $(x_T, y_T)$ and a receiver at 
$(x_R, y_R)$. We can express $\theta_i, \theta_s, r_i, r_s$ as follows:

\begin{align*}
   r_s &= \sqrt{y_R^2 + (x_R - x)^2} \\
   \cos \theta_s &= \frac{y_R}{r_s} \\
   r_i &= \sqrt{y_T^2 + (x - x_T)^2} \\
   \cos \theta_i &= \frac{y_T}{r_i}
\end{align*}
Plugging these into \eqref{eq:fieldMagEq1} we get
\begin{equation}
   d(|E_s|^2) = \frac{S^2 \eta_0 P_i \ y_T y_R }{2 \pi (y_R^2 + (x_R - x)^2) ( y_T^2 + (x -
   x_T)^2)} \ dx
   \label{eq:localScaPow2dSetupParam}
\end{equation}
We can also express $x$ in terms of $t$ by inverting the equation
\[ (ct) = \sqrt{y_T^2 + (x - x_T)^2} + \sqrt{y_R^2 + (x_R - x)^2}. \] 
In order to invert this, we need to square twice to get rid of any square roots.
However, in order for the $x^4$ and $x^3$ terms to cancel, we need to make a further
coordinate transformation so the origin is at the specular point of reflection: 
\begin{equation}
   x \to x' = x -x_T - y_T \left( \frac{x_R - x_T}{y_R + y_T} \right)
   \label{eq:coordTrans}
\end{equation}
%}}}
\newpage
%}}}
   
\end{document}
% Bibliography {{{
\bibliographystyle{IEEEtran}
\bibliography{refs}
\end{document}
%}}}
