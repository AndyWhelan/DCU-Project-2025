% LaTeX Boilerplate
% LaTeX Boilerplate:    Packages {{{
\documentclass{article}
\usepackage{algorithmic}
\usepackage{algorithm}
\usepackage{amsmath,amsfonts}
\usepackage{array}
\usepackage{arydshln} % for horizontal dashed lines in table
\usepackage{balance}
\usepackage{chngcntr}
\usepackage{cite}
\usepackage{gensymb} % for degrees symbol
\usepackage{graphicx}
\usepackage{mathtools}
\usepackage{setspace}
\usepackage{stfloats}
\usepackage[caption=false,font=normalsize,labelfont=sf,textfont=sf]{subfig}
\usepackage{tabularx}
\usepackage{textcomp}
\usepackage{tikz}
\usepackage{titlesec}
\usetikzlibrary{calc}
\usetikzlibrary{math}
\usepackage{url}
\usepackage{verbatim}
\usepackage{xcolor}
\usepackage{hyperref}
%}}}
% LaTeX Boilerplate:    Environment Setup  {{{
\definecolor{customBlue}{RGB}{0,80,150}
\hypersetup{
   colorlinks=true,
   linkcolor=customBlue,
}
\setlength{\parindent}{0pt}
\setlength{\parskip}{1em}
\counterwithin{equation}{section}
% }}}
% Title {{{
\titleformat{\subsection}
   {\normalfont\normalsize\bfseries} % format
   {\thesubsection}                % label
   {0.8em}                           % sep
   {}                              % before-code
\titlespacing*{\subsection}
   {0pt}
   {3.5ex}
   {1.0ex}
\begin{document}
\renewcommand{\thesection}{\Alph{section}}
\begin{titlepage}
   \centering
   \vspace*{\fill}  % Pushes content to vertical center

   {\Huge\bfseries Effective Roughness Models \par}
   \vspace{1cm}

   {\Large Andrew Whelan \par}
   \vspace{0.5cm}

   {\large \date{\today} \par}

   \vspace*{\fill}  % Pushes remaining space below
\end{titlepage}
% }}}

% TODO {{{
\section*{TODO}
The below derivations are not quite as nice as I'd like, however, I'm probably
spending a little too much time on prettifying the assumptions so... here's a list of
things to do after I've looked at the actual model comparisons and gotten some
results (focusing on that this week):
\begin{enumerate}
   \item Refine the notation,
   \item Make a glossary,
   \item Complete the assumptions,
   \item Come up with a sensible taxonomy of assumptions, and format appropriately
   \item Refine the derivations so that we only number significant results,
\end{enumerate}
% }}}
% TODO: revise equation numbering later on, to reflect "common" assumptions across
% scattering models

\newpage
\setcounter{section}{14} % "N"
\setcounter{subsection}{-1}
\setcounter{equation}{0}
\section*{(N) Notation}
The following generic notation is used throughout:
\begin{align}
   & \mathcal{E} = \text{Re}\{ \mathbf{E} e^{j \omega t } \} \ \text{is the electric field (}
      \mathbf{E} \text{ is the phasor).}
      \label{eq:eFieldDef} \\
   & \mathcal{H} = \text{Re}\{ \mathbf{H} e^{j \omega t } \} \ \text{is the magnetic field (}
      \mathbf{H} \text{ is the phasor).}
      \label{eq:hFieldDef} \\
   & \mathbf{E} = \vec{E} e^{\phi_E}
      \label{eq:ePhasorDef} \\
   & \mathbf{H} = \vec{H} e^{\phi_H}
      \label{eq:hPhasorDef} \\
   & \vec{E} \cdot \vec{H} = | \vec{E} | | \vec{H} | \cos
      \theta_{\mathcal{S}}
      \label{eq:spatialAngleDef} \\
   & \mathcal{S} \overset{\text{def}}{=} \mathcal{E} \times \mathcal{H} \ \text{is the
      Poynting vector.}
      \label{eq:poyntingDef} \\
   & \phi_{\mathcal{S}} = \phi_E - \phi_H \\
   & Z \overset{\text{def}}{=} \frac{|\mathbf{E}|}{|\mathbf{H}|} \ \text{is the EM-wave impedance}
      \label{eq:impedanceDef} \\
   & | \langle \mathcal{ S } \rangle | \overset{\text{def}}{=} \left| \frac{\omega}{2
      \pi} \int_{0}^{\frac{2 \pi}{\omega}} S(t) dt \right| = \ldots =
      \frac{1}{2Z} |\mathbf{E}|^2 \sin (\theta_{\mathcal{S}}) \cos (
      \phi_{\mathcal{S}} )
      \label{eq:poyntingAvgDef}
\end{align}

\setcounter{section}{6} % "F"
\setcounter{subsection}{-1}
\setcounter{equation}{0}
\section*{(F) Field Assumptions}
The Rx and Tx are not in the near-field (a reasonable assumption in an outdoor
scenario), so we have
   % 2. E_s-incoherence in Rx-region {{{
   %\subsection*{$E_s$-incoherence in Rx-region}
\begin{equation}
   \theta_{\mathcal{S}} = \frac{\pi}{2}
   \label{eq:spatialAngleAssumption}
\end{equation}
\begin{equation}
   \phi_{\mathcal{S}} = 0
   \label{eq:phaseAngleAssumption}
\end{equation}
The wave is travelling in free space until it hits an obstruction, so
\begin{equation}
   Z = \eta_0
   \label{eq:freeSpaceAssumption}
\end{equation}
      The contributions to $E_s$ arriving at the Rx have random, uncorrelated phases
      \begin{equation}
         \implies |E_{s}|^2 = \int_X d(|E_s|^2), \ \text{where}
         \label{eq:incoherence}
      \end{equation}
      \begin{itemize}
         \item $d(|E_s|^2)$ is the squared-field contribution from $dX$ over
         \item $X$, the wall surface.
      \end{itemize}
   % }}}
\newpage
\setcounter{section}{1}   % Because first section is unnumbered
\setcounter{subsection}{-1}
\setcounter{equation}{-1}  % ```
\section*{(A) \ Core Assumptions}
\counterwithin{equation}{subsection}
% A.0.1 Field Decomposition {{{
\subsection{Generic Setup}
\subsubsection{Field Decomposition}
The total electric field phasor at the boundary of a wall $X$ ($E_{\partial X}$) can
be decomposed into incident, transmitted, specularly-reflected, and
diffusely-scattered component fields:
\begin{equation}
   E_{\partial X} = E_i + E_t + E_r + E_s.
   \label{eq:fieldDecomposition}
\end{equation}
% }}}
% A.0.2. Field Continuity {{{
\subsubsection{Field Continuity}
\begin{align}
   &E_i, E_t, E_r, E_s \in C^0( \mathbb{R}^n \times \mathbb{C} ), \ \text{where} \ n
      \leq 3. 
   \label{eq:fieldContinuity}
\end{align}
\vspace{-\belowdisplayskip} 
\vspace{-\parskip}
% }}}
% A1. Constant Local Power Balance {{{
\counterwithout{equation}{subsection}
\counterwithin{equation}{section}
\setcounter{equation}{0}
\subsection{Local Power Balance}
$S_i$ is the phasor Poynting vector for the incoming wave. So, we have % TODO: move to notation section


Letting $P_z$ denote the total power associated with a given field component, $dP_z$
denote the total power associated with the portion incident on $dX$, and 
\begin{align}
   dP_i &= dP_t + dP_r + dP_s \nonumber \\
   S_i \nonumber \\
   \implies \frac{|E_i|^2}{2 \eta_0}dX_i &= P_t + \frac{R^2 |\Gamma|^2
   |E_i|^2 d \Omega_i r_i^2}{} \nonumber
\end{align}
A constant fraction ($S^2$) of the total power incident on $dX$ ($dP_i$) is converted
into diffusely scattered power ($dP_s$): 
\begin{equation}
   dP_s = S^2 dP_i.
   \label{eq:sDef}
\end{equation}
% }}}
   % A2. Scattering Pattern {{{
      %TODO: keeping here for ref. Remove later
      %\begin{align}
      %   dP_s = \int_{\Omega_X} I( \Omega ) d \Omega_s &= K \cos \theta_s d \Omega_s,
      %      \label{eq:lambertianLawGeneric} \\
      %   &= \left( \frac{\cos \theta_s dP_s }{ \int_{\Omega_X} \cos \theta_s' d 
      %      \Omega_s' } \right) d \Omega_s, \ \ \text{where} \nonumber
      %\end{align}
\subsection{Scattering Pattern $\mathfrak{S}$} % TODO: replace with assumption on scattering pattern
$\mathfrak{S}$ is 
The radiant intensity $I$ is given by a constant times a scattering pattern $K \cdot
\mathfrak{S}$. This relates to the above as follows:
\begin{align}
   dP_s &= \int_{\Omega_X} K\mathfrak{S}( \Omega ) d \Omega_s \\
   &= \int_{\Omega_X}\left( \frac{\mathfrak{S}(\Omega) dP_s }{ \int_{\Omega_X}
      \mathfrak{S}(\Omega_s') d \Omega_s' } \right) d \Omega_s, \ \ \text{where}
      \nonumber 
\end{align}
\begin{itemize}
   \item $\mathfrak{S}$ is the scattering pattern,
   \item $K$ is a normalization constant relating to the maximum intensity,
   \item $\Omega$ is the set of angular coordinates,
   \item $\Omega_X$ is the scattering domain. Formally, $\Omega_X = \{ \Omega \
      | \ E_s(r, \Omega) \neq 0 \}$, and
   \item $d \Omega_s$ is the solid angle of the scattered ray-tube from $dX$.
\end{itemize}
      % }}}


\renewcommand{\thesection}{\arabic{section}}
%\counterwithout{equation}{section}
\setcounter{section}{0}
\setcounter{equation}{0}
\newpage
\section{2D Lambertian-Model}
Let 
\begin{itemize}
   \item $d \theta_{Rx}$ be the Rx aperture,
   \item $d l_{Rx} = r_s d \theta_{Rx}$ be the Rx aperture length, and
   \item $d P_{Rx} = \frac{d(|E_s|^2)}{2 \eta_0} dl_{Rx}$ be the power through the Rx
      aperture.
\end{itemize}
\begin{align}
   \implies d(|E_s|^2) &= \frac{2 \eta_0 }{dl_{Rx}} dP_{Rx} \nonumber \\
   &= \frac{2 \eta_0 }{dl_{Rx}} \int_{\theta_{s} - \frac{d
      \theta_{Rx}}{2}}^{\theta_{s} + \frac{d \theta_{Rx}}{2}} I( \theta_s ) d \theta
      \nonumber \\
   &\overset{\mathclap{d \theta_{Rx} \to 0}}{\approx} \ \ \ \ \frac{2 \eta_0
      I(\theta_s) d \theta_{Rx}}{dl_{Rx}} = \frac{2 \eta_0 d \theta_{Rx}}{dl_{Rx}}
      I(\theta_s) \nonumber \\
   &\overset{\mathclap{}}{=} \ \ \ \frac{2 \eta_0
      d \theta_{Rx}}{dl_{Rx}} \left( \frac{\cos \theta_s dP_s}{ \int_{\Omega_s} \cos
      \theta_s' d \Omega_s' } \right) \nonumber \\ 
   &= \frac{2 \eta_0 \cos \theta_s d \theta_{Rx} d P_s}{dl_{Rx}
      \int_{-\frac{\pi}{2}}^{\frac{\pi}{2}} \cos \theta_s' d \theta_s'} = 2 \eta_0 
      \cos \theta_{s} \frac{ d \theta_{Rx} }{2 dl_{Rx} } d P_s = \frac{\cos
      \theta_s}{r_s} \eta_0 dP_s \nonumber \\
   &\overset{\mathclap{\eqref{eq:sDef}}}{=} \ \ \ \frac{S^2 \cos \theta_s}{r_s}
      \eta_0 dP_i = \frac{S^2 \cos \theta_s}{r_s} \eta_0\frac{|E_i|^2}{2 \eta_0} \cos
      \theta_i dx \nonumber \\ 
   &= \frac{S^2 \cos \theta_i \cos \theta_s}{2r_s} |E_i|^2 dx
   \label{eq:fieldPre}
\end{align}
A far-field cylindrical incoming wave allows us to assume 
\begin{equation}
   |E_i|^2 = \frac{A}{r_i}
   \label{eq:farField2d}
\end{equation}
Then, combining \eqref{eq:fieldPre} and \eqref{eq:farField2d} with the
expression for power, we get
\begin{align}
   P_i &= \int_0^{2 \pi} \frac{|E_i|^2}{2 \eta_0} r_i d \theta \nonumber \\
   &= \frac{A}{2 \eta_0} \int_0^{2 \pi} d \theta \nonumber \\ 
   &= \frac{A \pi}{\eta_0} \nonumber \\
   \implies A &= \frac{\eta_0 P_i}{\pi} \nonumber \\
   \implies |E_i|^2 &= \frac{\eta_o P_i}{\pi} \frac{1}{r_i} \nonumber \\
   \implies d(|E_s|^2) &= \frac{S^2 \eta_0 P_i \cos \theta_i \cos \theta_s }{2 \pi
      r_s r_i} \ dx
   \label{eq:fieldMagEq1}
\end{align}

\subsection*{Coordinate Transformations}
Our setup consists of a transmitter at $(x_T, y_T)$ and a receiver at 
$(x_R, y_R)$. We can express $\theta_i, \theta_s, r_i, r_s$ as follows:

\begin{align*}
   r_s &= \sqrt{y_R^2 + (x_R - x)^2} \\
   \cos \theta_s &= \frac{y_R}{r_s} \\
   r_i &= \sqrt{y_T^2 + (x - x_T)^2} \\
   \cos \theta_i &= \frac{y_T}{r_i}
\end{align*}
Plugging these into \eqref{eq:fieldMagEq1} we get
\begin{equation}
   d(|E_s|^2) = \frac{S^2 \eta_0 P_i \ y_T y_R }{2 \pi (y_R^2 + (x_R - x)^2) ( y_T^2
      + (x - x_T)^2)} \ dx
   \label{eq:localScaPow2dSetupParam}
\end{equation}
We can also express $x$ in terms of $t$ by inverting the equation
\begin{equation}
   (ct) = \sqrt{y_T^2 + (x - x_T)^2} + \sqrt{y_R^2 + (x_R - x)^2}
   \label{eq:pathLength}
\end{equation} 
In order to invert this, we need to square twice to get rid of any square roots.
However, in order for the $x^4$ and $x^3$ terms to cancel, we need to make a further
coordinate transformation so the origin is at the specular point of reflection: 
\begin{align}
   x \to x' &= x -x_T - y_T \left( \frac{x_R - x_T}{y_R + y_T} \right) \nonumber \\ 
   \implies x' &= x - \left( \frac{x_T y_R + x_R y_T}{y_R + y_T} \right)
   \label{eq:coordTrans}
\end{align}
For the Tx term $(x - x_T)$:
\begin{align}
   x - x_T &= x' + \frac{y_T(x_R - x_T)}{y_R + y_T}
\end{align}
For the Rx term $(x_R - x)$:
\begin{align}
   x_R - x &= -x' + x_R - \left(\frac{x_T y_R + x_R y_T}{y_R + y_T}\right) \nonumber
      \\
   &= -x' + \frac{x_R(y_R+y_T) - (x_T y_R + x_R y_T)}{y_R+y_T} \nonumber \\
   \implies x_R - x &= -x' + \frac{y_R(x_R - x_T)}{y_R + y_T}
\end{align}

Rewriting \eqref{eq:pathLength} with these new expressions:
\begin{equation*}
   ct = \sqrt{y_T^2 + \left(x' + \frac{y_T(x_R-x_T)}{y_R+y_T}\right)^2} + \sqrt{y_R^2
      + \left(\frac{y_R(x_R-x_T)}{y_R+y_T} - x'\right)^2}
\end{equation*}

To simplify, define the distances from the transmitter and receiver to the specular
point on the surface:
\begin{align*}
   R_{T0} &:= \sqrt{y_T^2 + \left(\frac{y_T(x_R-x_T)}{y_R+y_T}\right)^2} \\
   R_{R0} &:= \sqrt{y_R^2 + \left(\frac{y_R(x_R-x_T)}{y_R+y_T}\right)^2}
\end{align*}
We then get 
\begin{equation*}
   ct = \sqrt{R_{T0}^2 + 2x' \frac{y_T(x_R-x_T)}{y_R+y_T} + (x')^2} + \sqrt{R_{R0}^2
      - 2x' \frac{y_R(x_R-x_T)}{y_R+y_T} + (x')^2}
\end{equation*}

Isolate the second square root and square both sides:
\begin{align*}
   &(ct)^2 - 2ct \sqrt{R_{T0}^2 + 2x' \frac{y_T(x_R-x_T)}{y_R+y_T} + (x')^2} +
      \left(R_{T0}^2 + 2x' \frac{y_T(x_R-x_T)}{y_R+y_T} + (x')^2\right) \\
   &= R_{R0}^2 - 2x' \frac{y_R(x_R-x_T)}{y_R+y_T} + (x')^2
\end{align*}

Now collect terms and cancel:
\begin{align*}
   2x'(x_R-x_T) + (ct)^2 + R_{T0}^2 - R_{R0}^2 &= 2ct \sqrt{R_{T0}^2 + 2x'
      \frac{y_T(x_R-x_T)}{y_R+y_T} + (x')^2}
\end{align*}

Squaring both sides: 
\begin{align*}
   &4\left( (x_R - x_T)^2 + (ct)^2 \right)^2 (x')^2 + \\
   &4(x_R - x_T) \left( R_{T0}^2 - R_{R0}^2 + (ct)^2 \left( \frac{y_R - y_T}{y_R +
      y_T} \right) \right) x' + \\ 
   & \left( (ct)^2 + R_{T0}^2 - R_{R0}^2 \right)^2 - 4(ct)^2 R_{T0}^2
\end{align*}

Finally, we can solve for $x$ :
\begin{align} 
   x &= \frac{x_T y_R + x_R y_T}{y_R + y_T} - \frac{b}{2a} \pm \frac{\sqrt{b^2 -
      4ad}}{2a}, \ \text{where} \\
   a &= 4\left( (x_R - x_T)^2 + (ct)^2 \right)^2 \nonumber \\
   b &= 4(x_R - x_T) \left( R_{T0}^2 - R_{R0}^2 + (ct)^2 \left( \frac{y_R - y_T}{y_R +
      y_T} \right) \right), \nonumber \\
   d &= \left( (ct)^2 + R_{T0}^2 - R_{R0}^2 \right)^2 - 4(ct)^2 R_{T0}^2 \nonumber \\
   R^2_{T0} &= y_T^2 + \left(\frac{y_T(x_R-x_T)}{y_R+y_T}\right)^2, \ \text{and}
      \nonumber \\ 
   R^2_{R0} &= y_R^2 + \left(\frac{y_R(x_R-x_T)}{y_R+y_T}\right)^2 \nonumber
\end{align}
\newpage
\section{Plane Wave Incident on PEC}
\subsection{Questions for Conor for Fri 20 Jun 2025}
\begin{enumerate}
   \item Discussion of the above work, any thoughts on further refinements apart from
      what I've listed already in the TODO section (I've spent too long on this part
      but it will be nice to have for a writeup. I'm going to focus on the MATLAB
      code now)?
   \item In the above I've done the derivation for the power assuming a cylindrical
      incoming wave. However, in the original MATLAB code setup you seem to have
      assumed the Tx is in the far-region (hence a plane wave approximation). Do you
      think it's sensible to adjust the above from \eqref{eq:fieldPre} for $|E_i|$?
   \item What measures specifically do you think would be good to compare first? I
      think last time you mentioned the power profiles as a first step. So, do you
      think calibrating the model correctly 
\end{enumerate}
\end{document}
% Bibliography {{{
\bibliographystyle{IEEEtran}
\bibliography{refs}
\end{document}
%}}}
