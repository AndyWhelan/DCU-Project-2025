% LaTeX Boilerplate
% LaTeX Boilerplate:    Packages {{{
\documentclass{article}
\usepackage{algorithmic}
\usepackage{algorithm}
\usepackage{amsmath,amsfonts}
\usepackage{array}
\usepackage{arydshln} % for horizontal dashed lines in table
\usepackage{balance}
\usepackage{cite}
\usepackage{gensymb} % for degrees symbol
\usepackage{graphicx}
\usepackage{stfloats}
\usepackage[caption=false,font=normalsize,labelfont=sf,textfont=sf]{subfig}
\usepackage{tabularx}
\usepackage{textcomp}
\usepackage{tikz}
\usetikzlibrary{calc}
\usetikzlibrary{math}
\usepackage{url}
\usepackage{verbatim}
\usepackage{xcolor}
\setlength{\parindent}{0pt}
\setlength{\parskip}{1em}
%}}}
\begin{document}
In what follows, $E$ refers to the phasor (spatial, or time-averaged) component of
the electric field.
% ---Models---
% Models:               ER1 {{{
\section{ER Model} 
\subsection*{Assumptions}
\begin{itemize}
   \item \textbf{$E$-field decomposition}:
      \begin{equation}
         E = E_i + E_r + E_s + E_t, \ \text{where}
         \label{eq:eDecomp}
      \end{equation}
   \begin{itemize}
       \item $E$ is the total electric field,
       \item $E_i$ is the \textbf{i}ncident field, 
       \item $E_r$ is the specularly \textbf{r}eflected component of the scattered field,
       \item $E_s$ is the diffusely \textbf{s}cattered component of the scattered field, and
       \item $E_t$ is the \textbf{t}ransmitted component of the scattered field. 
   \end{itemize}
   \item \textbf{Lambertian local diffuse scattering}: 
   \begin{equation}
      | dE_s | = K \sqrt{\cos \theta_s}, \ \text{where} 
      \label{eq:lambertianLocalDiffuse}
   \end{equation}
   \begin{itemize}
      \item $dE_s$ is the field contribution at the Rx due to an infinitesimal
         wall surface element $dW$,
      \item $\theta_s$ is the scattering angle relative to the wall normal, and
      \item $K$ is the maximum field magnitude (when the component is normal to the
         wall)
   \end{itemize}
\item \textbf{Local power balance ($S$ parameter):}
      \begin{equation}
         S^2 := \frac{ dP_s }{ dP_i } \ \text{is assumed to be constant for a
         given wall},
         \label{eq:sDef}
      \end{equation}
   \item \textbf{Rx-region wave incoherence:} 
      \begin{equation}
         | E_{s} |^2 = \int_W | dE_s |^2 \ dW, \ \text{due to random phases.}
         \label{eq:incoherence}
      \end{equation}
\end{itemize}
\subsection*{Local Scattering Power Balance}
In all ER model setups we have the following general formula:
\begin{equation}
   S^2 |dE_i|^2 d A_i = \int_{A_s} |dE_s|^2 d A_s,
   \label{eq:localScaPowBal}
\end{equation}
where $A_i$ and $A_s$ denote the area (or length, if in 2d) of the incident and
scattered ray-tubes.
\subsection*{2D-Setup}

\subsubsection*{$|E_i|$ as a function of $P_i$ and $r_i$}
A far-field cylindrical incoming wave allows us to assume 
\begin{equation}
   |E_i| = \frac{A}{\sqrt{r_i}}
   \label{eq:farField2d}
\end{equation}
Denoting the time-averaged incoming Poynting vector by $\Pi_i$, we get 
\begin{align*}
   \Pi_i &= \frac{|E_i|^2}{2 \eta_0} \\
   \implies P_i &= \int_{0}^{2\pi} \Pi_i r d \theta \\ 
   &= \frac{A^2}{2 \eta_0} \int_0^{2 \pi} d \theta \\ 
   &= \frac{A^2}{\pi \eta_0} \\
   \implies A &= \sqrt{\frac{\eta_0 P_i}{\pi}}
\end{align*}
\begin{equation}
   \implies |E_i| = \sqrt{ \frac{\eta_o P_i}{\pi} } \frac{1}{\sqrt{r_i}}
   \label{eq:2dIntensityExpr}
\end{equation}
\subsubsection*{Local Field Intensity}
In 2d, the surface element simplifies to a line element:
\[ dW \to dx \]
and the area reduces to a length:
\[ d A \to d l = \cos \theta dx \]
So the local scattering power-balance \eqref{eq:localScaPowBal} becomes
\begin{align*}
   S^2|dE_i|^2 \cos \theta_i dx  &= \int_{-\frac{\pi}{2}}^{\frac{\pi}{2}} K^2 r_s \cos
   \theta_s d \theta_s  \\
   \implies K^2 &= \frac{S^2|dE_i|^2 \cos \theta_i dx}{r_s
   \int_{-\frac{\pi}{2}}^{\frac{\pi}{2}} \cos \theta_s d \theta_s } \\
   \implies K &= S |dE_i| \sqrt{ \frac{\cos \theta_i dx}{2r_s} } \\
   \implies |dE_s| &= S \sqrt{ \frac{\cos \theta_i \cos \theta_s dx}{2r_s} } |dE_i|
\end{align*}
Combining this with \eqref{eq:2dIntensityExpr} we get
\begin{equation}
   \implies |dE_s|^2 = \frac{ S^2 \eta_0 P_i \cos \theta_i \cos \theta_s dx}{2
   \pi r_s r_i }
   \label{eq:localScaPow2d}
\end{equation}

\subsubsection*{Coordinate Transformations}
Our setup consists of a transmitter at $(x_i, y_i)$ and a receiver at 
$(x_s, y_s)$. We can express $\theta_i, \theta_s, r_i, r_s$ as follows:

\begin{align*}
   r_s &= \sqrt{y_s^2 + (x_s - x)^2} \\
   \cos \theta_s &= \frac{y_s}{r_s} \\
   r_i &= \sqrt{y_i^2 + (x - x_i)^2} \\
   \cos \theta_i &= \frac{y_i}{r_i}
\end{align*}
Plugging these into \eqref{eq:localScaPow2d} we get
\begin{equation}
   |dE_s|^2 = \frac{S^2 \eta_0 P_i \ y_i y_s }{2 \pi (y_s^2 + (x_s - x)^2) ( y_i^2 + (x -
   x_i)^2)} dx
   \label{eq:localScaPow2dSetupParam}
\end{equation}
%}}}
\newpage
%}}}
   
\end{document}
% Bibliography {{{
\bibliographystyle{IEEEtran}
\bibliography{refs}
\end{document}
%}}}
